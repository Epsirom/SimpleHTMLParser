\documentclass[a4paper]{article}
%\documentclass[twocolumn]{article}

\usepackage{graphicx}
\usepackage{listings}
\usepackage{xcolor}
%\usepackage{enumitem}
\usepackage{enumerate}
\usepackage{CJKutf8} %注意这里用的是CJKutf8而不是CJK
\usepackage{indentfirst}

\usepackage{tikz} % 画流程图用的
\usepackage{qtree}
\usepackage{indentfirst}%英文首行缩进
\usepackage{fancyhdr} % 排版格式
\usepackage{hyphenat} % 单词断字
\usepackage{amsmath} % for {aligned}, 公式换行
\usepackage{multicol}% 多栏排版
\usepackage{balance}% 双栏最后一页对齐
\usepackage{subfigure}% 多图
\usepackage{booktabs}% 表格画线,\toprule, \midrule, \bottomrule
\usepackage{ulem}
%\usepackage{clrscode}

% in texlive-science
\usepackage{algorithm}
\usepackage{algpseudocode}% an improvement from algorithmicx for algorithmic

\usepackage{setspace}
 
\usepackage[utf8]{inputenc}


\usepackage{xspace}

%======= XXX 要编译两遍才能有标签和引用等效果 =====%
 
\usepackage[top=2.54cm,bottom=2.54cm,left=3.17cm,right=3.17cm]{geometry} % a4paper standard
\usepackage[unicode=true]{hyperref} %注意这里不能加CJKbookmarks=true,否则会乱码
\usetikzlibrary{arrows,decorations.pathmorphing,backgrounds,positioning,fit,automata,trees}

\hypersetup{
    pdfauthor={ouoline},
    %pdftitle={test},
    %pdfsubject={Subject},
    %pdfkeywords={Keyword1, Keyword2, ...},
    %pdfcreator={LaTeX with hyperref package},
    %pdfproducer = {dvips + ps2pdf},
    %bookmarksnumbered=true,
    %colorlinks=no,
    pdfborder={0 0 0},
    %bookmarksopen=true,
}
%------------------------------------------------------------------%
 
\setcounter{secnumdepth}{5} % 编号的深度,4 表示到 paragraph 一级
%\setcounter{tocdepth}{4} % 目录中的深度
 
%------------------------------------------------------------------%
\usepackage{color}
\definecolor{lightgray}{rgb}{.9,.9,.9}
\definecolor{darkgray}{rgb}{.4,.4,.4}
\definecolor{purple}{rgb}{0.65, 0.12, 0.82}

\lstdefinelanguage{JavaScript}{
  keywords={typeof, new, true, false, catch, function, return, null, catch, switch, var, if, in, while, do, else, case, break},
  keywordstyle=\color{blue}\bfseries,
  ndkeywords={class, export, boolean, throw, implements, import, this},
  ndkeywordstyle=\color{darkgray}\bfseries,
  identifierstyle=\color{black},
  sensitive=false,
  comment=[l]{//},
  morecomment=[s]{/*}{*/},
  commentstyle=\color{purple}\ttfamily,
  stringstyle=\color{red}\ttfamily,
  morestring=[b]',
  morestring=[b]"
}

\lstset{% general command to set parameter(s)
        language={C},
        %numbers=left,
        basicstyle=\tt, % 默认对所有字符使用等宽字体
        keywordstyle=\color{blue},%\bfseries\underbar% underlined bold black keywords
        identifierstyle=,           % nothing happens
        stringstyle=\color{purple},
        commentstyle=\color{gray},
        escapechar=`,
        showstringspaces=false,     % no special string spaces
        breaklines=true, % 自动换行
    }


%------------------------------------------------------------------%
 
% 自己定义新命令,参数依次是
% \newcommand{新命令名称(带反斜线)}[参数个数(最多9个)]{命令定义}
% 实际上相当于宏替换
% \newcommand{\sayhelloto}[1]{hello,#1}
\newcommand{\template}[3] {
    \item \textbf{#1}
    \begin{enumerate}
        \item[\textbf{#2}] #3 
    \end{enumerate}
}
\newcommand{\jie}[2]{\template{#1}{解}{#2}}
\newcommand{\zheng}[2]{\template{#1}{证明}{#2}}
 
%------------------------------------------------------------------%
 
% 放在导言区,设置全局行距
\linespread{1.6}
 
% 放在导言区,公式编号和章节相关
%\makeatletter % `@' now normal ``letter''
%\@addtoreset{equation}{section}
%\makeatother % `@' is restored as ``non-letter''
%\renewcommand\theequation{\oldstylenums{\thesection}%
%.\oldstylenums{\arabic{equation}}}


\begin{document}

\begin{CJK*}{UTF8}{gbsn}
    \CJKindent
    \setlength{\parindent}{2em} % no indent
 
    \pagestyle{fancy}
  
    %\begin{center}
    %\Huge{title}
    %\vspace{25pt} % 25pt between title and text
    %\end{center}
    \title{\huge{计算机与网络体系结构(2)}\\\Large{编译原理}\\{\large Project 2:文法分析器的实现}}
    \author{软件11\hspace{10pt}陈华榕\hspace{10pt}2011013236}
    \date{\today}
    \maketitle
    \tableofcontents    
    \newpage

    \section{实验背景}
    \subsection{实验环境}
    如需复现实验结果请准备一台PC机并安装Windows/OS X/Linux等桌面系统,然后安装Chrome等高级浏览器。

    \subsection{目录结构}
{
\tikzstyle{every node}=[draw=black,thick,anchor=west]
\tikzstyle{selected}=[draw=red,fill=red!30]
\tikzstyle{optional}=[dashed,fill=gray!50]
\begin{tikzpicture}[%
  grow via three points={one child at (0.5,-0.7) and
  two children at (0.5,-0.7) and (0.5,-1.4)},
  edge from parent path={(\tikzparentnode.south) |- (\tikzchildnode.west)}]
  \node {2011013236}
    child { node [optional] {src $|$ 源码}
        child { node [selected] {exp2.html $|$ 请用Chrome打开}}
        child { node {exp2.js $|$ 在所给框架基础上修改的结果}}
    }
                child [missing] {}
                child [missing] {}
    child { node [optional] {data $|$ 测试数据}
        child { node {test.html $|$ 原本提供的测试数据}}
        child { node {baidu.html $|$ 真实的百度个性首页HTML源码}}
        child { node {google.html $|$ 用Google搜索“钱康来”的页面HTML源码}}
    }
                child [missing] {}
                child [missing] {}
                child [missing] {}
    child { node {report.pdf $|$ 本报告}}
    child { node {README.txt $|$ 请先阅读此文件}}
;
\end{tikzpicture}
}

    \subsection{完成情况}
    \label{sec:comp}
    在助教所给代码框架基础上,通过完善代码文件exp2.js,完成了符合实验要求的文法分析器,能进行HTML文法分析(除了不支持无关闭标签外,基本支持了大部分HTML文法规则)。
    \par 为说明实验成果的可靠性,我还提供了来自百度个性首页和Google搜索结果页中的HTML源码作为文法分析器的输入(将其中所有无关闭标签,包括img、input、hr等,都人工进行了关闭处理),得到的输出结果说明文法分析器具备一定的处理真实数据的能力。
    \par 如果您关心本实验的进度情况,可查看\href{https://github.com/Epsirom/SimpleHTMLParser}{https://github.com/Epsirom/SimpleHTMLParser}。不过本次实验比较简单,也没啥进度可看的……

    \section{实验分析}
    \subsection{整体思路}
    首先,需要进行文法分析,将输入的HTML标签及其属性、标签间的关系等都分析出来,在内存中依照HTML结构构建一棵树(根节点为document)。
    \par 然后,遍历树,遇到base标签或a标签,相应进行处理,提取出其中的链接。比如a标签,需要处理其href属性;而base标签,可能需要处理其href属性,也可能是处理其文本子节点。
    \par 最后,处理提取出的链接,可能有三种情况:相对地址、站点内绝对地址、绝对地址。考虑更强的鲁棒性,三种情况都需要借助栈进行“.”目录及“..”目录的分析。

    \subsection{理论分析}
    \subsubsection{文法与分析程序}
    HTML分析文法为:
    \begin{eqnarray*}
        tag&\rightarrow&comment|dtype|TEXT|node|tag~tag|\epsilon\\
        comment&\rightarrow&<!--~TEXT~-->\\
        dtype&\rightarrow&<!~DOCTYPE~TEXT>\\
        node&\rightarrow&<~ID~attr~/>\\
        node&\rightarrow&<~ID~attr~>~tag~</~ID~>\\
        attr&\rightarrow&ID~=~TEXT~attr\\
        attr&\rightarrow&ID~attr\\
        attr&\rightarrow&\epsilon
    \end{eqnarray*}
    \par 根据此文法可以发现,该文法并不是LL(1)文法,不过,我们可以将其改写为LL(1)文法,如下:
    \begin{eqnarray*}
        tag&\rightarrow&<!--~TEXT~-->~|~<!~DOCTYPE~TEXT>~|~TEXT~|~<~ID~attr~nodend~|~\epsilon\\
        nodend&\rightarrow&/>~|~>~tags~</~ID~>\\
        tags&\rightarrow&tag~tags~|~\epsilon\\
        attr&\rightarrow&ID~attrend~|~\epsilon\\
        attrend&\rightarrow&attr~|~=~TEXT~attr
    \end{eqnarray*}
    \par 据此写出文法分析程序如下:
    \begin{lstlisting}
void parseTag() {
    switch (lookahead) {
        case "<!--":
            MatchToken("<!--");
            MatchToken(TEXT);
            MatchToken("-->");
            break;
        case "<!":
            MatchToken("<!");
\end{lstlisting}\clearpage
\begin{lstlisting}
            MatchToken("DOCTYPE");
            MatchToken(TEXT);
            MatchToken(">");
            break;
        case "<":
            MatchToken("<");
            MatchToken(ID);
            parseAttr();
            parseNodeEnd();
            break;
        case epsilon:
            break;
        default:
            exit(error);
    }
}
void parseNodeEnd() {
    switch (lookahead) {
        case "/>":
            MatchToken("/>");
            break;
        case ">":
            MatchToken(">");
            parseTags();
            MatchToken("</");
            MatchToken(ID);
            MatchToken(">");
            break;
        default:
            exit(error);
    }
}
void parseTags() {
    switch (lookahead) {
        case "<!--":
        case "<!":
\end{lstlisting}\clearpage
\begin{lstlisting}
        case "<":
            parseTag();
            parseTags();
            break;
        case "</":
            break;
        default:
            exit(error);
    }
}
void parseAttr() {
    switch (lookahead) {
        case ID:
            MatchToken(ID);
            parseAttrEnd();
            break;
        case "/>":
        case ">":
            break;
        default:
            exit(error);
    }
}
void parseAttrEnd() {
    switch (lookahead) {
        case ID:
            parseAttr();
            break;
        case "=":
            MatchToken("=");
            MatchToken(TEXT);
            parseAttr();
            break;
        default:
            exit(error);
    }
\end{lstlisting}\clearpage
\begin{lstlisting}
}
    \end{lstlisting}

    \subsubsection{链接的类型}
    实验要求中明确提到要给出使用的自动机及其状态图,但我觉得本实验不需要通过自动机进行分析。为切合实验要求,就用自动机来判断链接的类型好了。也就是通过自动机判断一个链接是相对地址、站点内绝对地址、绝对地址。在代码中的实现,是通过正则表达式来进行判断的。
    \par 这里假定所给链接都是合法的。
    \par 
    \begin{tikzpicture}[>=stealth',shorten >=1pt,auto,node distance=4cm]
      \node[initial,state,accepting]  (p0)                          {相对地址};
      \node[state,accepting]          (p1)        [right of=p0]     {相对地址};
      \node[state,accepting]          (p2)         [above of=p1]       {站内绝对地址};
      \node[state,accepting](p3)         [right of=p2]        {相对地址};
      \node[state,accepting](p4)    [right of=p1]   {绝对地址};


      \path[->] (p0)        edge                    node    {$/$}    (p2)
                            edge                    node    {not in $\{/,:\}$}  (p1)
                (p1)        edge    [loop below]    node    {not in $\{/,:\}$}   (p1)
                            edge                    node    {$/$}   (p3)
                            edge                    node    {$:$}   (p4)
                (p2)        edge    [loop above]    node    {all}   (p2)
                (p3)        edge    [loop above]    node    {all}   (p3)
                (p4)        edge    [loop below]    node    {all}   (p4);
    \end{tikzpicture}

    \subsection{实现分析}
    \subsubsection{对exp2.js的改动}
    \noindent\begin{tabular}
        {c|l}
        \hline
        原行号 & 改动说明\\
        \hline
        51 & lookAhead参数为string的情形,直接逐字符比较。\\
        \hline
        109 & 分析tag的属性,不断调用parseAttr并将结果记录直至分析不出attr为止。\\
        \hline
        113 & 分析拥有子节点的节点,分析子节点并加入children,直至分析到原节点的关闭标识。\\
        \hline
        117 & 在此本应将children释放,但结合调用相关的代码,发现这里其实啥也不做也没问题。\\
        \hline
        157 & 分析attr,之前代码已分析出等号,接着就通过正则表达式匹配出具体的value即可,\\
        &若是非法输入就抛出异常。\\
        & 提交的代码支持两种value:一种是用单引号或双引号引起来的字符串;\\&另一种是没有引号的字符串。\\
        \hline
        263 & 根据分析出的a标签寻找其href属性,直接在attrs里找到href即可。\\
        \hline
        268 & 遇到base标签,处理其子文本节点,存到baseHref即可。\\
        \hline
        271 & 遇到base标签,处理其属性,得到的href属性存到baseHref即可。\\
        \hline
        290 & 得到了所有href属性,在此将其全部转化成绝对地址的表示。\\
        \hline
    \end{tabular}

    \subsubsection{href转化成绝对地址}
    上述原290行处将得到的所有href属性都转化成绝对地址的表示,在此讨论一下实现方法。
    \par 实际上通过栈就能简单地实现了。
    \par 首先,计算保留部分,即站点的根地址。
    \par 然后,将href按“$/$”分隔后得到输入队列,根据不同的href类型构建不同的初始栈:
    \begin{itemize}
        \item 相对地址:以baseHref的“$/$”分隔结果作为初始栈。
        \item 绝对地址:以空栈作为初始栈。
        \item 站内绝对地址:以baseHref的“$/$”分隔结果的保留部分作为初始栈。
    \end{itemize}
    \par 接着,依次处理队列中的每个字符串:
    \begin{itemize}
        \item “.”:忽略。
        \item “..”:若栈长度大于保留部分长度(即栈中存在非保留部分),就进行一次出栈操作,否则忽略。
        \item 其他:将该字符串压栈。
    \end{itemize}
    \par 最后,将栈中剩下的字符串按栈底到栈顶的顺序以“$/$”连接起来,就得到了转化成绝对地址的新href。

    \subsection{如何测试}
    用Chrome打开src/exp2.html,在textarea中输入,点击“分析”按钮得到结果。
    \par 为方便测试,提供了三个数据文件,放于data文件夹中,可用文本编辑器打开它们并将内容完整地复制到textarea中,即可分析出结果。数据文件的来源见\ref{sec:comp}。
    \par baidu.html和google.html都被我人为地加了base标签:
    \begin{itemize}
        \item Baidu:http://www.baidu.com/a/b/c/d/e
        \item Google:https://www.google.com.hk/search/qiankanglai/nb
    \end{itemize}

    \section{实验结果}
    能如愿地分析出输入文本中的所有链接,并都以绝对地址的方式输出。输出结果正确,且鲁棒性较好,能处理真实世界的数据。

\clearpage
\end{CJK*}
\end{document}


